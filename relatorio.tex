\documentclass[12pt, letterpaper]{article}
\usepackage[utf8]{inputenc}
\usepackage{mathtools}
\usepackage{indentfirst}

\title{Erros}
\author{Grupo 39}
\date{October 9, 2022}

\begin{document}

\maketitle

\section*{Introdução}

Este trabalho prático visa estudar os erros de cálculo em números reais, que tendem a ser um grande problema para computadores, e em geral, em dispositivos capazes de avaliar o campo da aritmétrica.

Durante este trabalho irá ser estudado a forma como os valores aproximados, os limites, as somas infinitas e os erros absolutos afetam os valores calculados. 

Pretende-se assim analizar a eficácia do computador e dos próprios métodos usados, e consequentemente, os resultados obtidos via tais computações.

Para além disso, vale a pena frisar que todos os cálculos feitos e todos os valores foram obtidos usando a linguagem \textbf{C/C++}. 

\section*{Epsilon Máquina (macheps)}
O \textbf{macheps} utilizado pela máquina pode ser obtido pela função:
\begin{verbatim}
    type macheps() {
        type eps = 0.5;
        type macheps;
        while ((1+eps) != 1) {
            macheps = eps;
            eps /= 2;
        }
        return macheps;
    }
\end{verbatim}

$type \in (float, double, long double)$

Sendo assim temos 3 valores que o macheps vai tomar (dependendo do tipo a ser usado):

\begin{description}
    \item[Float:]\textit{macheps} = 1.192093\textit{e}-07
    \item[Double:]\textit{macheps} = 2.220446\textit{e}-16
    \item[Long Double:]\textit{macheps} = 1.084202\textit{e}-19
\end{description}

\small Nota: C só representa 7 algarismos significativos para estes valores, o que pode já ser uma causa de erros futuros.

\section*{Exercício 2}
Queremos calcular o valor aproximado de:
\[S=18\sum_{k=1}^{\infty}\frac{k!^2}{k^2(2k)!}\]

Podemos observar que \textit{S} é uma série de termos positivos e também sabemos que $\forall n$:
$$a_n = \frac{n!^2}{n^2(2n)!}$$
\newline

Passamos a calcular o limite entre $a_n$ e $a_{n+1}$ para saber se a série é convergente ou não. Sendo assim:
\end{document}