\documentclass[12pt, letterpaper]{article}
\usepackage[utf8]{inputenc}
\usepackage{mathtools}
\usepackage{indentfirst}

\title{Erros}
\author{Grupo 39}
\date{October 9, 2022}

\documentclass[12pt, letterpaper]{article}
\usepackage[utf8]{inputenc}
\usepackage{mathtools}
\usepackage{indentfirst}

\title{Erros}
\author{Grupo 39}
\date{October 9, 2022}

\begin{document}

\maketitle

\section*{Introdução}

Este trabalho prático visa estudar os erros de cálculo em números reais, que tendem a ser um grande problema para computadores, e em geral, em dispositivos capazes de avaliar o campo da aritmétrica.

Durante este trabalho irá ser estudado a forma como os valores aproximados, os limites, as somas infinitas e os erros absolutos afetam os valores calculados. 

Pretende-se assim analizar a eficácia do computador e dos próprios métodos usados, e consequentemente, os resultados obtidos via tais computações.

Para além disso, vale a pena frisar que todos os cálculos feitos e todos os valores foram obtidos usando a linguagem \textbf{C/C++}. 

\section*{Epsilon Máquina (macheps)}
O \textbf{macheps} utilizado pela máquina pode ser obtido pela função:
\begin{verbatim}
    type macheps() {
        type eps = 0.5;
        type macheps;
        while ((1+eps) != 1) {
            macheps = eps;
            eps /= 2;
        }
        return macheps;
    }
\end{verbatim}

$type \in (float, double, long double)$

Sendo assim temos 3 valores que o macheps vai tomar (dependendo do tipo a ser usado):

\begin{description}
    \item[Float:]\textit{macheps} = 1.192093\textit{e}-07
    \item[Double:]\textit{macheps} = 2.220446\textit{e}-16
    \item[Long Double:]\textit{macheps} = 1.084202\textit{e}-19
\end{description}

\small Nota: C só representa 7 algarismos significativos para estes valores, o que pode já ser uma causa de erros futuros.

\section*{Exercício 2}
Queremos calcular o valor aproximado de:
\[S=18\sum_{k=1}^{\infty}\frac{k!^2}{k^2(2k)!}\]

Podemos observar que \textit{S} é uma série de termos positivos e também sabemos que $\forall n$:
$$a_n = \frac{n!^2}{n^2(2n)!}$$
\newline

Passamos a calcular o limite entre $a_n$ e $a_{n+1}$ para saber se a série é convergente ou não. Sendo assim: 
$$\lim_{n \to \infty} \frac{a_{n+1}}{a_n}$$
\newline

Manipulando a expressão $a_{n+1}$ sabemos que:
$$a_{n+1} = \frac{(n+1)!^2}{(n+1)^2(2(n+1))!} = \frac{n!^2(n+1)^2}{(n+1)^2(2n+2)!} = \frac{n!^2}{(2n+2)!} = \frac{n!^2}{(2n+2)(2n+1)(2n)!} =$$
$$\frac{n!^2}{n^2(2n)!} \frac{n^2}{(2n+2)(2n+1)} = a_n\frac{n^2}{(2n+2)(2n+1)}$$
\newline

Assim, o limite entre $a_n$ e $a_{n+1}$ é:
$$\lim_{n \to \infty} \frac{a_{n+1}}{a_n} = \lim_{n \to \infty} \frac{a_n \frac{n^2}{(2n+2)(2n+1)}}{a_n} = \lim_{n \to \infty} \frac{n^2}{(2n+2)(2n+1)} = \lim_{n \to \infty} \frac{n^2}{4n^2+6n+2} = $$
$$\lim_{n \to \infty} \frac{n^2}{n^2(4+ \frac{6}{n} + \frac{2}{n^2})} = \frac{1}{4} < 1$$
\newline

Como o limite entre $a_n$ e $a_{n+1}$ é menor do que 1, pelo critério de D'Alembert, a série é convergente e o erro é majorado por $R_n \leq a_{n+1} \frac{1}{1-L}$. Portanto, basta encontrar um \textit{n} tal que $a_{n+1} \frac{1}{1-L} < \epsilon$.
\newline
Seja $S_n=18\sum_{k=1}^{n-1} \frac{k!^2}{k^2(2k)!}$ a soma parcial dos \textit{n} primeiros termos de \textit{S} e seja $R_n = S - S_n$. Tem-se que:
$$|R_n| \leq a_n \frac{1}{1-\frac{1}{4}} = \frac{4}{3} a_n$$
\newline
Logo, basta encontrar um \textit{n} tal que $\frac{4}{3} \frac{n!^2}{n^2(2n)!} < \epsilon$ e somar os primeiros \textit{n} termos da série. O programa usado foi o seguinte:
\begin{verbatim}
    // Somatório desde low até high
    double calc_sum(double low, double high) {
        double sum = 0;
        while (low <= high) {
            sum += s(low);
            low++;
        }
        return sum;
    }

    // função responsável pelo calculo de S num determinado n
    double s(double n) {
        return (18*pow(factorial(n),2))/(pow(n, 2)*factorial(2*n));
    }

    // Avalia valores para todos os intervalos de erros epsilon
    void error_values() {
        double epsilon[8] = {8,9,10,11,12,13,14,15};
        double n = 1;
        double prev = 1;
        double sum = 0;
    
        for (int i = 0; i < 8; i++) {
            while (s(n+1)*(4/3) >= pow(10,-epsilon[i])) n++;
            sum += calc_sum(prev, n);
            prev = n+1;
            printf("exp:%.0f | n:%.0f | sum:%0.19f | error:%0.19f\n", 
                epsilon[i], n, sum, pow(M_PI,2)-sum);
        }
    }
\end{verbatim}

Sabemos que o valor exato de \textit{S} é $\pi^2$. Podemos calcular o valor efetivo do erro absoluto através da expressão $E = |\pi^2 - S|$ para cada valor aproximado de \textit{S}. Para isso, tomamos o valor exato de $\pi$ disponível na biblioteca math.h do C. Como esse valor tem 20 casas decimais corretas, o erro absoluto é, no máximo, $5*10^{-21}$.

Resultados obtidos:
\begin{center}
    \begin{tabular}{|| c | c | c | c ||}       \hline
 $\epsilon$ & n & S & $|\pi^2 - S|$\\  [0.4ex] \hline\hline
 $10^{-8}$  & 13 & 9.8696043981327523653 & 0.0000000029566056270 \\  \hline
 $10^{-9}$  & 14 & 9.8696044004219967150 & 0.0000000006673612774 \\  \hline
 $10^{-10}$ & 16 & 9.8696044010547119285 & 0.0000000000346460638 \\  \hline
 $10^{-11}$ & 17 & 9.8696044010814016900 & 0.0000000000079563023 \\  \hline
 $10^{-12}$ & 19 & 9.8696044010889334430 & 0.0000000000004245493 \\  \hline
 $10^{-13}$ & 20 & 9.8696044010892602927 & 0.0000000000000976996 \\  \hline
 $10^{-14}$ & 22 & 9.8696044010893544396 & 0.0000000000000035527 \\  \hline
 $10^{-15}$ & 23 & 9.8696044010893579923 & 0.0000000000000000000 \\ [0.4ex]\hline
    \end{tabular}
\end{center}

Analisando os valores da tabela, os termos a somar foram-se mantendo relativamente baixos, mesmo para valores de $\epsilon$ pequenos. As aproximações obtidas pelo programa mostram que a série converge monotonamente para o valor exato de $\pi^2$, logo estão de acordo com o esperado.
Como é possível ver na última coluna da tabela, os valores dos erros absolutos são todos menores do que o valor de $\epsilon$ dado em cada caso. Logo, foram conseguidas aproximações corretas de \textit{S} para cada valor de $\epsilon$. Além disso, podemos ver que o erro absoluto é $10^{-16}$ e, portanto, menor que todos os $\epsilon$ dados.

\section*{Exercício 3}
Queremos estudar o comportamento das seguintes fórmulas para um cálculo aproximado da constante de Euler \textit{e}:

$$\textit{e} = \lim_{n \to \infty} (1+\frac{1}{n})^n, \textit{e} = \sum_{k=0}^{\infty} \frac{1}{k!}$$

Para a primeira fórmula \textbf{A.} temos que \textit{n} toma valores: \textit{n} = $10^j$, \textit{j} = 1,2,...,16

Para a segunda fórmula \textbf{B.} temos que \textit{m} toma valores: \textit{m} = 1,2,...20 
\newline

De forma a ser mais conviniente, em ambas fórmulas foram calculados 20 parcelas de valores, sendo a seguinte tabela o registo dos resultados obtidos:

\begin{center}
    \begin{tabular}{|| c|c|c ||} \hline
        \textit{j/m} & \textbf{A}. & \textbf{B}. \\ [0.4ex] \hline \hline
         0 & 2.0000000000000000000 & 1.0000000000000000000\\ \hline
         1 & 2.5937424601000018676 & 2.0000000000000000000\\ \hline
         2 & 2.7048138294215289257 & 2.5000000000000000000\\ \hline
         3 & 2.7181459268243561844 & 2.6666666666666665186\\ \hline
         4 & 2.7181459268243561844 & 2.7083333333333330373\\ \hline
         5 & 2.7182682371975284141 & 2.7166666666666663410\\ \hline
         6 & 2.7182804691564275146 & 2.7180555555555554470\\ \hline
         7 & 2.7182816939803724487 & 2.7182539682539683668\\ \hline
         8 & 2.7182817863957975391 & 2.7182787698412700372\\ \hline
         9 & 2.7182820308145094756 & 2.7182815255731922477\\ \hline
         10 & 2.7182820532347875542 & 2.7182818011463845131\\ \hline
         11 & 2.7182820533571101507 & 2.7182818261984929009\\ \hline
         12 & 2.7185234960372377522 & 2.7182818282861687109\\ \hline
         13 & 2.7161100340869008818 & 2.7182818284467593628\\ \hline
         14 & 2.7161100340870230063 & 2.7182818284582301871\\ \hline
         15 & 3.0350352065492618436 & 2.7182818284589949087\\ \hline
         16 & 1.0000000000000000000 & 2.7182818284590428703\\ \hline
         17 & 1.0000000000000000000 & 2.7182818284590455349\\ \hline
         18 & 1.0000000000000000000 & 2.7182818284590455349\\ \hline
         19 & 1.0000000000000000000 & 2.7182818284590455349\\ \hline
         20 & 1.0000000000000000000 & 2.7182818284590455349\\ [0.4ex] \hline
    \end{tabular}
\end{center}

A partir da tabela acima podemos observar três pontos importantes nos valores apresentados:

\begin{enumerate}
    \item Geralmente para valores "pequenos" de \textit{n} as aproximações tendem a ter valores muito longínquos dos valores que queremos obter (\textbf{A.} tende a aproximar mais rápido para valores mais baixos, mas vem com um defeito mais à frente falado.);
    \item Em geral a fórmula \textbf{B.} mostra-se ser mais consistente no aproximamento do valor de Euler;
    \item Para $n >= 13$ a fórmula \textbf{A.} começa a ter comportamento atípico, chegando mesmo a deixar de ser uma fórmula de cálculo viável a partir de $n >= 16$, onde passa apenas a gerar 1.0 como valor de cálculo, independente do valor de n. Isto é causado pelo facto de que no \textbf{C/C++}, é primeiro arredondado o valor de $1+\frac{1}{n}$ antes de se fazer a exponencial, pelo que para $n >= 10^{16}$ esse arredondamento torna-se 1.0000000000000000000, e assim sabendo, $1^n = 1$ independente do valor de n, sabemos também o porquê da estranheza dos valores obtidos.
\end{enumerate}

\section*{Conclusão}
Erros e arredondamentos irão ser sempre um poblema elementar em dispositvos capazes de fazer aritmétrica, pois por exemplo, um computador nunca será capaz de representar um número infinito (como o $\pi$) por mais memória que tenha, sendo apenas capaz de representar o valor como uma aproximação do valor real.
Em suma, este trabalho estudou os possíveis problemas que podem ser causados por tais erros e arredondamentos, tendo em conta um ponto de vista computacional.

\section*{}
\textit{Repositório com todo o material produzido para o trabalho: \href{https://github.com/Hasuru/AN_tp1}{Github}}

\end{document}
